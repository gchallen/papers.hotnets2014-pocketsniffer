\section{Related Work}
\label{sec:related}

Using client side measurements to either monitor or reconfigure
wireless networks has received a lot of attention in the recent
years. Mishra \textit{et al.}  proposed to collect client-side
measurements called \textit{site-reports}, which contain the client's
visibility to near-by APs or devices. This information is then used to determine AP channel
assignment~\cite{mishra:mccr2005,dasilva:mswim2008,mishra:mobicom2006},
or joint channel assignment and terminal
association~\cite{mishra:infocom2006}. Sen \textit{et al.} proposed to
use mobile devices to measure and improve the performance of wide-area
cellular networks~\cite{sen2011can}. Finally, the DARPA
RadioMap~\cite{radiomap} project aims to provide real-time awareness of
radio spectrum using idle radios on user devices.

Our proposed approach differs from existing ones in the following
aspects. First, we identify smartphones as an ideal vantage point for
both long-term network monitoring and short-term network
reconfiguration, due to their \textit{always on } and \textit{mostly
  idle} nature. Second, we distinguish between active and passive
clients and propose to only collect measurements from passive clients,
and exploit device proximity detection to jointly optimize network
performance for both. Third, we consider incentives for clients to
provide measurement data and measurement validation mechanisms, which
are missing in previous works.
