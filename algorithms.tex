\section{Coordination Scenarios}
\label{sec-algorithms}

When performing network adaptation \PS{} is designed to support multiple
different coordination patterns between clients sharing the same WLAN and
between overlapping WLANs. In all cases \PS{} algorithms utilize crowdsourced
measurements from inactive smartphones to attempt to improve performance for
active client by controlling AP channel assignments, client associations, and
AP power levels and transmission rates. Depending on the scenario, clients
may behave selfishly or provide incorrect data in hopes of either avoiding
the energy overhead of performing measurements or improving their own network
performance at the expense of others.

We are using \PS{} to explore several different CANSAS coordination
algorithms including maximization and game-theoretic approaches designed to
enable cooperation in each of the following common scenarios. Because we are
still determining which algorithms work best at achieving the objectives
appropriate to each scenario, we focus the discussion below on describing the
objective and the associated challenges.

\subsection{Single Network, Cooperative Clients}

In the first case a single network serves a set of clients which are
cooperative in the sense that they are willing to work together to achieve a
single common objective. Thus, \PS{} can assume that clients are willing to
provide truthful measurements. Typical home Wifi networks serving multiple
mobile devices fall into this category. Because clients are cooperative and a
global objective is shared, this scenario lends itself to the simplest
coordination algorithms. One example uses \PS{} measurements to select a set
of network parameters maximizing the aggregate throughput of all clients; a
variant prioritizes interactive sessions by only maximizing throughput to
interactive clients.

\subsection{Single Network, Selfish Clients}

In the second case a single network serves a set of clients each of which
wishes to maximize its own performance. Typical enterprise Wifi networks fall
into this category. This scenario presents two new complications compared
with the previous one. First, we must formulate a notion of social utility
balancing both performance and fairness. Second, \PS{} cannot assume that
clients are willing to provide truthful measurements. They may intentionally
mislead the system to try to improve their own performance at the expense of
other clients, or attempt to avoid the energy overhead of performing
measurements. We are addressing these challenges in two ways, both by
designing mechanisms that incentivize clients to perform accurate
measurements and by utilizing \PS{}'s control of the network APs to validate
client measurements.

\subsection{Multiple Networks, Cooperative Clients}

In the third case multiple independently-administered networks serve
clients that will cooperate within each network. Thus, \PS{} can assume that
clients provide truthful measurements but that each network acts in a
self-interested fashion. Overlapping home Wifi networks each serving multiple
mobile devices fall into this category. We are attempting to address this
scenario by formalizing the problem as a noncooperative game between the
competing networks where each attempts to selfishly maximize one of the local
objectives described in the first scenario.

\subsection{Multiple Networks, Selfish Clients}

In the final case multiple independently-administered networks serve sets of
self-interested clients. Thus, \PS{} can neither assume that clients will
provide truthful measurements nor that the networks will not act selfishly.
Overlapping enterprise Wifi networks fall into this category. Because \PS{}
must arrange cooperation both between clients within each network and between
the overlapping networks, this represents the most challenging scenario. We
are attempting to address it by framing it as a two-level noncooperative game
where coordination occurs first between the networks and then between
clients.
