\section{Other Forms of Reuse}
\label{sec-other}

The combination of low price and many capabilities makes discarded phones
suitable for many other forms of reuse. With their GPS chips, discarded
smartphones can easily replace dedicated car GPS navigation units costing
hundreds of dollars. By utilizing their on-board storage, discarded
smartphones could be a part of a personal storage cloud by serving as powered
storage lockers, caching media and shared files close to the user's primary
device at home, at work, or while in the car. Programmable thermostats that
look much like smartphones cost hundreds of dollars, despite the fact that
smartphones have temperature sensors and Wifi and represent a much cheaper
option. And discarded iPhones are already getting second lives as security
cameras through the popular Presence app~\cite{presence-peoplepower}.

Another option is to use discarded phones to create storage clusters similar
to the FAWN~\cite{fawn} fast array of wimpy nodes. Discarded smartphones have
many similarities in terms of processing power, memory, and storage with the
devices used by FAWN, combined with a considerable advantage in price, and
could be used to create Fast Arrays of Discarded Smartphones (FADS).

\section{Conclusions}
\label{sec-conclusion}

To conclude, we believe that the millions of discarded smartphones represent
a significant opportunity, and can be used in many cases as replacements for
sensor nodes in our effort to better instrument the world around us. Despite
not being designed for ultra-low power consumption, smartphones can be
operated efficiently enough to be able to operate continously while
harvesting energy. We are beginning outdoor experiments with commodity solar
panels and plan to use our nodes to gather data to support the Neon citizen
science project. Through intelligent reuse, we can turn techno trash into
treasure.

\section*{Acknowledgments}

Philippe Bonnet and Aslak Johansen provided helpful feedback on this paper.
We thank Darren Beck and Blake Hosmer for discussing Sprint's sustainability
efforts with us.
