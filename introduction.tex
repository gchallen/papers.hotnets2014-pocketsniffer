\section{Introduction}

The rapid proliferation of smartphones represents both a challenge and an
opportunity for wireless spectrum allocation. On one hand, smartphones are
competing for the same limited spectrum already crowded with other devices.
On the other hand, the growing network of mobile, always-on smartphones
constitutes an unprecedented new source of detailed measurements of spectrum
usage and availability. In addition, because smartphones are always on but
idle much of the time, they are in a unique position to provide measurements
about \textit{other} nearby active wireless devices---such as laptops,
tablets, or other smartphones---meaning that measurements can be performed
passively without disturbing devices actively using the spectrum, a
capability that has not been exploited by other systems designed to
incorporate client-side feedback into spectrum allocation. This type of
measurement also improves on existing approaches that use periodic planned
site surveys, since the data they provide is neither continuous nor
representative of the experience of actual network users.

Figure~\ref{fig-example} illustrates the type of spectrum adaptation
smartphones can enable. As a user engages with their laptop, their smartphone
is nearby and capable of helping the laptop improve its usage of the
spectrum, in this case by scanning other channels and determining that both
the client and access point should move to a different, less-congested
portion of the spectrum. As a second example, as one user engages with their
smartphone, the smartphone of a second user sitting nearby may be inactive
and able to perform measurements. Smartphones can also provide continuous
observations of wireless performance to help providers provision and maintain
their networks. We refer to these approaches as \textbf{c}rowdsourcing
\textbf{a}ccess \textbf{n}etwork \textbf{s}pectrum \textbf{a}llocation using
\textbf{s}martphones, or \textbf{CANSAS}.

Realizing CANSAS requires novel integration between smartphones, an adaptive
network, and algorithms enabling cooperative spectrum allocation on both
short and long timescales. This paper describes a prototype system
implementing CANSAS for Wifi networks called \textbf{PocketSniffer}.
PocketSniffer uses a publish-subscribe architecture to perform both
asynchronous and synchronous measurement collection from passive smartphones
and use them to improve the performance of active clients. Measurements are
used as inputs to new algorithms---including game-theoretic approaches---that
perform joint channel assignment, power control, rate selection, and
transmission scheduling to improve spectrum efficiency assuming a variety of
different network structures, including both fully-cooperative settings and
cases where multiple independently-administered networks compete for the
local spectrum resources. These scenarios are representative of both typical
home and corporate Wifi deployments. PocketSniffer measurements are also
provided to network administrators in order to perform long-term monitoring
and network maintenance. PocketSniffer provides a platform for testing these
new algorithms and determining how to incentivize client measurement
collection, assuring that collected measurements are accurate, and minimizing
the energy overhead for participating devices.
