\section{Introduction}

\sloppypar{The rapid proliferation of smartphones creates both challenges and
new opportunities for wireless networks. On one hand, smartphones compete for
the same limited spectrum already crowded with other devices. On the other
hand, because smartphones are \textit{always on} but \textit{mostly idle},
they are ideal for observing other nearby active wireless devices---such as
laptops, tablets, or other smartphones. When used for continuous network
adaptation, offloading measurements from active to inactive clients allows
data collection to avoid disturbing active sessions, a capability that has
not been adequately exploited by other systems using client-side
feedback~\cite{vasan:infocom2005,mishra:mccr2005,dasilva:mswim2008,mishra:infocom2006,mishra:sigmetrics2006,mishra:mobicom2006,murty:hotnets2008,rayanchu:mobicom2011}.
When used for network monitoring and debugging, smartphones provide more
valuable measurements than planned site surveys, since the data that
smartphones provide is continuous and representative of wireless conditions
experienced by users while surveys are neither. We refer to these approaches
collectively as \textbf{c}rowdsourcing \textbf{a}ccess \textbf{n}etwork
\textbf{s}pectrum \textbf{a}llocation using \textbf{s}martphones, or
\textbf{CANSAS}.}

Realizing CANSAS requires novel integration between smartphones and an
adaptive network, along with new algorithms enabling cooperative spectrum
allocation on both short and long timescales. This paper describes a
prototype system implementing CANSAS for Wifi networks called \PS{}.
Implemented as a smartphone app, \PS{} is straightforward to deploy. It uses
a pub-sub architecture to collect data from passive smartphones and uses
measurements to improve network performance. \PS{} also captures the large
number of measurements made naturally by smartphones as they discover and
connect to networks, valuable data that is currently discarded.

To enable short-term adaptation, measurements are triggered by and used as
inputs to new algorithms that can alter channel assignments, control AP power
levels and rate selection, and alter client associations in order to improve
network performance and allocate available spectrum more effectively. \PS{}
allows different algorithms to be deployed to support a variety of different
network structures, including both fully-cooperative settings and cases where
multiple networks overlap and compete for the local spectrum resources,
scenarios representative of both typical home and enterprise Wifi
deployments. To enable long-term adaptation measurements are provided to
network administrators in order to perform network monitoring, maintenance,
and capacity planning. Realizing \PS{}, however, requires addressing a set of
open research challenges: determining how to incentivize client measurements,
ensuring that client measurements are accurate, ensuring fairness in the
energy consumed by the measurement process, and dealing with differences in
smartphone wireless measurement capabilities.

The rest of this paper is structured as follows. Section~\ref{sec-design}
describes the design of \PS{} while Section~\ref{sec-algorithms} describes
algorithms using \PS{} data to perform cooperative network adaptation to
support several common deployment scenarios. We continue by presenting
results from a prototype \PS{} system in Section~\ref{sec-results} before
discussing open challenges in Section~\ref{sec-challenges} and concluding in
Section~\ref{sec-conclusions}.
