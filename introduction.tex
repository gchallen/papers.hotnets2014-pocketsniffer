\section{Introduction}

The rapid proliferation of smartphones creates both challenges and new
opportunities for wireless networks. On one hand, smartphones compete for the
same limited spectrum already crowded with other devices. On the other hand,
the growing network of mobile, always-on smartphones constitutes an
unprecedented new source of detailed network measurements. Because
smartphones are \textit{always on} but \textit{mostly idle}, they are ideal
for observing other nearby active wireless devices---such as laptops,
tablets, or other smartphones. When used for continuous network adaptation,
offloading measurements from active to inactive clients allows data
collection to avoid disturbing active sessions, a capability that has not
been exploited by other systems using client-side feedback. When used for
network monitoring and debugging, measurements from mobile smartphones
provide more valuable data than approaches such as planned site surveys,
since the data that smartphones provide is continuous and representative of
the wireless conditions experienced by actual network users while surveys are
neither. We refer to these approaches collectively as \textbf{c}rowdsourcing
\textbf{a}ccess \textbf{n}etwork \textbf{s}pectrum \textbf{a}llocation using
\textbf{s}martphones, or \textbf{CANSAS}.

\sloppypar{Realizing CANSAS requires novel integration between smartphones,
an adaptive network, and algorithms enabling cooperative spectrum allocation
on both short and long timescales. This paper describes a prototype system
implementing CANSAS for Wifi networks called \PS{}. \PS{} uses a pub-sub
architecture to perform both asynchronous and synchronous data collection
from passive smartphones and use collected measurements to improve the
performance of active clients. Synchronous measurements are used as inputs to
new algorithms---including game-theoretic approaches---that can alter channel
assignments, control AP power levels and rate selection, and alter client
associations in order to improve spectrum efficiency. \PS{} allows different
algorithms to be deployed to support a variety of different network
structures, including both fully-cooperative settings and cases where
multiple independently-administered networks overlap and compete for the
local spectrum resources, scenarios representative of both typical home and
corporate Wifi deployments. Asynchronous measurements are provided to network
administrators in order to perform long-term monitoring, network maintenance,
and capacity planning. Realizing the \PS{} system, however, requires
addressing a set of open research challenges, including determining how to
incentivize client measurements, ensure that data provided by clients is
accurate, reduce and ensure fairness in the energy consumed by the
measurement process, and deal with significant variation in the measurement
capabilities across different smartphone models.

The rest of this paper is structured as follows. We begin by describing how
using passive smartphones improves on previous client-side measurement
approaches in Section~\ref{sec-related}. Section~\ref{sec-design} briefly
describes the design of the data collection portions of \PS{} while
Section~\ref{sec-algorithms} discusses several game-theoretic algorithms that
utilize \PS{} measurements to perform spectrum adaptation. We
continue by presenting results from a prototype \PS{} system in
Section~\ref{sec-results} before discussing open challenges in
Section~\ref{sec-challenges} and concluding in Section~\ref{sec-conclusions}.
