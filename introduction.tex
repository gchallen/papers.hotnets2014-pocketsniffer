\section{Introduction}

Smartphone technologies are advancing rapidly, bringing power into users
pockets that is changing the way we live. The rapid rate at which consumers
purchase new smartphones can be seen as primarily a response to the speed at
which this technology is improving. Short device lifetimes, while unfortunate
from a sustainability perspective, help support companies that build and sell
smartphone hardware and software. Unfortunately smartphones, like most other
electronics, are difficult to dispose of properly. Many end up unused in desk
drawers, discarded in landfills, or shipped to poor countries where they are
dangerously dismantled in an effort to extract precious metals.

Given smartphones' current role in bringing about transformative
technological change, it is hard to argue that consumers should hang on to
outdated devices in the name of sustainability. Instead, we believe it will
be more effective to focus on how to reuse the devices we currently discard.
There are three reasons why the time is right for this effort. First, unlike
previous generations of ``feature phones'', the smartphone market is
coalescing around a small set of platforms, with this homogeneity reducing
the burden of reusing discarded devices. Today, each phone in an electronics
recycling bin runs a different OS; in three years, half of the smartphones in
the same bin may run Android.

Second, current smartphones have an attractive feature set for many non-phone
applications: size and power requirements facilitating easy deployment;
microphones, cameras, and other sensors built-in; touch screens for
interacting with users. And the volume at which they are produced combined
with the rate at which consumers are replacing them produces an extremely
competitive price point for discarded devices given their capabilities.

Finally, smartphones are well-integrated into existing communication
infrastructures. They can transmit data using text messages, Wifi networks,
and high-speed mobile communication technologies like 3G. If Wifi is
available, no service plans are required to allow recycled smartphones to
become part of the ``Internet of Things''. And with carriers increasingly
interested in ``machine-to-machine'' applications~\cite{sprint-m2m}, we
expect to see increasing service flexibility allowing discarded devices to be
cheaply connected to pervasive mobile cellular and data networks.

To provide an idea of the potential of discarded devices, the U.S.
Environmental Protection Agency (EPA) estimates that 141~million mobile
devices became ready for end-of-life management in 2009, of which only 11.7
million (8\%) were collected for recycling~\cite{epa-ewasteweb}. The
129~million discarded phones are enough to place an average of \textit{200
phones} on all 600,000~bridges in the United States, or one phone every
\textit{2~feet} on every mile of the 46,876~mile US interstate system.

In this paper, we investigate reusing smartphones sensor network ``motes''.
Compared to motes, discarded smartphones have many advantages, which we
outline in Section~\ref{sec-comparison}. And while power consumption is a
concern and the discarded phone's major weakness, we show in
Section~\ref{sec-results} that a simple and unoptimized sense-and-send
application running on a Nexus~S phone can last over a week on a full battery
charge, even while preserving the familiar and powerful Android programming
environment. To begin, the next section reviews the current state of
smartphone sustainability.
