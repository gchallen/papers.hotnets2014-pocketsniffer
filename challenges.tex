\section{Open Challenges}
\label{sec-challenges}

Realizing the \PS{} system at scale requires research addressing several new
and existing challenges, each described in more detail below.

\subsection{Proximity Detection}

To estimate performance for active clients without disturbing their active
sessions \PS{} attempts to collect data from nearby inactive smartphones.
Unfortunately, energy-efficient physical proximity detection on smartphones
remains an open research problem~\cite{searchlight-mobicom12}, may disrupt
network performance for the clients involved, and may also not be sufficient
to determine how an active client will be affected by potential changes to
the network configuration. It may be more appropriate to utilize a
signal-based notion of proximity, alone or in combination with physical
proximity. We are designing \PS{} clients to be able to utilize multiple
notions of proximity when deciding whether to participate in synchronous
queries in order to determine the most effective approach.

\subsection{Measurement Availability and Energy Consumption}

As discussed earlier many smartphone Wifi chipsets do not allow detailed
measurements, either due to hardware, firmware, or driver limitations. Our
current prototype \PS{} implementation utilizes special hardware and future
versions may take advantage of modified firmware or drivers, but broader
adoption will require better support for channel measurements on many
smartphone devices. This may be more feasible than it sounds, given that
\PS{}'s detailed channel measurements require much less information than that
provided by full monitor mode packet scans, which are typically seen to
represent a potential security threat. Ideally, some form of channel
utilization estimation could be implemented directly in Wifi hardware devices
and activated as needed. Better hardware support would also provide an
opportunity to lower the energy overhead of performing detailed channel
measurements, another obstacle to deploying \PS{}.

However, because detailed channel utilization measurements will always
require the radio to be on continuously the energy consumption incurred will
always be greater than when the radio is disabled or power-cycled. Thus, we
are exploring several other ways to address this energy overhead by limiting
the amount of data clients need to provide. One way is to limit these
measurements to support synchronous adaptation by active client operated by
the same user: i.e., by Bob's smartphone only to help Bob's laptop locate a
better channel in the example above. In the second part of the previous
example, if Bob has forgotten his smartphone, Alice's device may be unwilling
to incur the battery drain necessary to help Bob and only agree to provide
scan results rather than detailed channel utilization measurements.

\subsection{Incentivizing Data Collection}

While measurements collected from \PS{} clients should improve network
performance, the overhead of performing measurements---particularly detailed
channel utilization measurements---requires that \PS{} incentivize
participation. Some users may not want to install the \PS{} app required to
collect data. To avoid this problem, it may be appropriate for \PS{} networks
with long-term registered users, such as enterprise networks, to require
users install the \PS{} app as part of the network registration process. The
authentication required by network providers in this scenario also helps
during synchronous queries by naturally identifying sets of related clients.
For each user operating a set of devices---attaching a smartphone, laptop,
and tablet to the \PS{} network---a single instance of the \PS{} app running
on a smartphone is sufficient to provide measurements allowing all of their
other devices to access the network.

For temporary sessions at wireless ``hotspots'', it may be more appropriate
for the network to suggest---but not require---installing the \PS{} app and
provide improved quality-of-service to clients who do so, essentially trading
measurements for bandwidth. Other providers that operate multiple networks
serving mobile clients, such as Boingo, may want to require measurements from
long-term subscribers or integrate \PS{} into their preexisting apps such as
the Boingo Wifi Finder~\cite{boingo-playstore-url}.

\subsection{Validating Measurements}
\label{subsec-validation}

Related to the problem of incentivizing installation is ensuring that clients
return accurate measurements. In most cases we believe that users will be
unwilling or unable to tamper with the \PS{} app in ways that would cause it
to return faulty measurements, but there is still the potential for more
sophisticated users to break the app to either try to improve their own
performance at the expense of other clients or avoid the energy overhead of
performing measurements at all by returning bogus data in response to \PS{}
queries. From the perspective of designing \PS{} to resist these behaviors we
do not distinguish between malicious and lazy clients. Instead, we focus on
designing measurement validation mechanisms that will identify both types of
misbehavior.

The easiest way for \PS{} to detect incorrect measurements is by manipulating
trusted network devices such as APs. As an example, to prevent clients from
returning falsified scan results, \PS{} APs include a random nonce in each
beacon message that clients must report allowing the network to verify that
the device actually heard the scan it is reporting. \PS{} networks may also
manipulate AP power levels to verify that these changes are reflected by
client measurements, to prevent clients from inaccurately reporting low AP
signal strengths in attempt to receive better service. Similar techniques can
be applied to verify detailed measurements, since the messages clients
overhear can be validated by network-controlled APs.

A second approach assuming that most clients will cooperate with \PS{} is to
compare measurements from several different devices. Without a large number
of co-located clients to compare against it may be difficult to immediately
identify false measurements, but over time noncooperative clients may be
identified using reputation mechanisms.
