\section{Open Challenges}
\label{sec-challenges}

\subsection{Implementing CANSAS Algorithms}

\PS{} allows the proposed CANSAS spectrum assignment algorithms
described in the next sections to be implemented at multiple places within
the WLAN by having agents subscribing to measurement streams generated by
\PS{} clients. For a small home WLAN, a programmable AP may subscribe
to the results of asynchronous queries as well as issue synchronous queries.
For a large infrastructure WLAN, programmable APs may perform local
adaptation by issuing synchronous queries while a central controller
configures all APs using measurements from asynchronous queries.

\XXXnote{GWA: Write more here...}

\subsection{Incentivizing Installation}

To avoid interfering with active terminals, CANSAS utilizes idle smartphones
to perform measurements about the channel conditions experienced by nearby
active terminals as well as performing exploration of other available
spectrum. To incentivize these measurements, SPs using \PS{} may
require or suggest that users install the \PS{} app. Requiring
terminals to submit measurements may be appropriate for SPs operating
corporate or campus WLANs where terminals are likely to repeatedly utilize
the network over an extended period of time, and where the authentication
typically required by the SP to access the network can naturally identify
relationships between terminals. These SPs may require installing the
\PS{} app as part of a network registration process and deny service
to terminals used by clients that fail to install the app or fail to continue
providing measurements. For each client operating a set of
terminals---attaching a smartphone, laptop, and tablet to the \PS{}
WLAN---a single instance of the \PS{} app running on their single
smartphone is sufficient to provide measurements allowing all of their other
terminals to access the WLAN.

For temporary sessions at wireless ``hotspots'', it may be more appropriate
for the SP to suggest---but not require---installing the \PS{} app
and provide improved quality-of-service to clients who do so, effectively
incentivizing clients by trading measurements for bandwidth. However, SPs
that operate multiple WLANs serving mobile clients, such as Boingo, may want
to require measurements from long-term subscriber or integrate \PS{}
into their preexisting apps such as the Boingo Wifi
Finder~\cite{boingo-playstore-url}.

\subsection{Validating Measurements}
\label{subsec-validation}

The CANSAS adaptation algorithms described next require accurate measurements
to achieve their goals. When apps compete for available spectrum, they may
have an incentive to try to manipulate the network using false measurements,
or may want to avoid the energy drain of performing measurements. When
multiple WLANs attempt to cooperate, they may have an incentive to provide
false data to each other. From the perspective of designing a network to
resist these behaviors we do not distinguish between malicious and lazy
terminals. Instead, we focus on designing \PS{} allowing validation
of terminal measurements that will identify both types of bad behavior. We
briefly describe several of these approaches below.

\subsubsection{Exploiting network control:\space} The easiest way for the
WLAN to detect incorrect measurements is by manipulating the
measurements using trusted devices such as APs controlled as part of
the WLAN. As an example, to prevent terminals from returning fake scan
information, \PS{} APs include a random nonce in each beacon
message which terminals must report as part of their scan results,
allowing the SP to verify that the terminal actually heard the scan it
is claiming to have measured. The WLAN may also manipulate the power
levels of APs to verify that these changes are reflected by terminal
measurements, to prevent the case where a terminal will incorrectly
report a low signal strength from an AP to try and force it to
increase its transmission power. Similar techniques can be applied to
verify detailed measurements, since the messages terminals overhear
can be validated by a APs controlled by the SP.

\subsubsection{Comparing terminal measurements:\space} A second approach
assuming that most terminals will cooperate with \PS{} is to compare
measurements from several different terminals. Without a large number of
co-located terminals to compare against it may be difficult to immediately
identify false measurements, but over time noncooperative terminals may be
identified through reputation mechanisms.

\subsubsection{Utilizing mutual observability:\space} Finally, when
considering cooperation between multiple SPs operating their own independent
WLANs we will explore the concept of \textit{mutual observability} as the
basis for establishing trust between self-interested SPs. Mutual
observability refers to the fact that although the SPs are operating
independent networks, there must be areas where their coverage overlap---if
not, there would be no need for them to coordinate spectrum allocations. This
means that there are events that should be visible to APs in both WLANs,
potentially providing a source of mutually-verifiable information to the SPs
as they attempt to cooperate.

\subsection{Controlling Client Energy Consumption}
\label{subsec-energy}

We are exploring several ways to address the energy overhead of performing
detailed spectrum utilization measurements. One is to limit these
measurements to support synchronous adaptation by active terminals operated
by the same client: i.e., by Bob's smartphone only to help Bob's laptop
locate a better channel in the example above. In the second part of the
example, if Bob has forgotten his smartphone, Alice's device may be unwilling
to incur the battery drain necessary to help Bob and only agree to provide
scan results rather than detailed channel utilization measurements. A second
approach is to ensure that the algorithms triggering synchronous data
collection only request detailed measurements when needed. Finally, the
\PS{} app includes an energy cap applied over each charging session
limiting the total energy clients will devote to \PS{} measurements.
Once this limit is exhausted, the client will not contribute measurements in
response to synchronous queries or perform detailed measurements.

