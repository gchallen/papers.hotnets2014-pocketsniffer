\section{PocketSniffer Design}
\label{sec-design}

This section describes the design of \PS{}, a system enabling CANSAS for Wifi
networks by providing the client-side measurements needed to enable the
coordination algorithms described in the next section. We begin with an
overview of the \PS{} system from the perspective of a network user.

\subsection{Overview}

When Alice and Bob register for their campus WLAN they are required to
install the \PS{} monitoring app on their smartphones. As they travel around
campus, their smartphones collect measurements on the health and performance
of the campus network based on queries established by network administrators,
reporting these measurements in an energy-neutral way by uploading data only
when their smartphones are charging. Network administrators can use this data
to identify poorly-served areas of campus, locate APs that are over- or
underutilized, and for other network monitoring and debugging purposes.

In addition, as Alice and Bob sit together in a campus cafe using the campus
WLAN to surf the web---Alice on her tablet, Bob on his smartphone---suddenly
a new source of interference begins to disrupt their network performance.
Unfortunately this interfering client cannot be overheard at the AP they
are associated with, but the AP can tell that Alice's and Bob's networking
environment has degraded. At this point, it triggers the \PS{} app on Alice's
inactive smartphone to search for a less-congested channel. Based on the
results \PS{} may move their active devices to a different channel, adjust
the AP power level, or suggest they associate with a different campus AP, all
without performing measurements on their active clients that could interrupt
their web browsing. Without collecting measurements directly from Bob's
smartphone, \PS{} is still able to exploit the proximity between Alice and
Bob and use Alice's nearby inactive smartphone to estimate the impact of
changes on Bob's active smartphone to ensure that any spectrum adaptation
benefits both devices.

Later, when Alice and Bob return home to neighboring apartments their
overlapping home Wifi networks use \PS{} measurements as inputs to perform
cooperative spectrum allocation. Despite the lack of centralized control,
their two networks jointly adapt to allocate spectrum in a way that improves
performance for both Alice and Bob.

\subsection{Performing Measurements}
\label{subsec-measurement}

\PS{} collects two types of measurements from clients---scan results and
spectrum utilization information---in two different ways---asynchronously and
synchronously.

\subsubsection{Measurement types}

Scan results are inexpensive to collect and provide a high-level view of the
network including visible APs and their signal strengths. Android smartphones
already perform Wifi scans at regular intervals, even while associated with a
Wifi network. Recovering this data incurs no energy overhead for clients as
long as the measurements are only uploaded while the battery is charging, and
our recent analysis of 89~million scan results collected from 139~smartphones
over 5~months~\cite{conext14-pocketsniffer} has demonstrated the value of
these measurements for network monitoring and debugging.

However, because smartphones are frequently idle, periodic scans may not
reflect either locations where users actually use their devices or locations
where their devices use the network. To better connect Wifi scans with
interactive usage and network activity, \PS{} annotates each scan with two
pieces of information: (1) whether the scan was performed during interactive
use as estimated by the screen state, and (2) the timestamp since device's last
data transfer, because not every interactive session includes Wifi usage and
because smartphones perform background data transfers with the screen
disabled.

In contrast, spectrum utilization measurements are expensive to collect and
may not be possible to collect on all clients, but provide a very detailed
view of spectrum usage. The ability of Wifi chipsets to observe link-layer
signaling traffic and packets sent by other devices varies from device to
device, and because these measurements require disabling the power-save mode
used by mobile Wifi chipsets they consume extra energy even if measurement
upload is performed in an energy-neutral way. We discuss several research
challenges caused by Wifi chipset heterogeneity in
Section~\ref{sec-challenges}.

\subsubsection{Query types}

Asynchronous measurements are used to perform network monitoring and as a
replacement for expensive site surveys in order to do spatial and temporal
spectral capacity planning. For asynchronous queries \PS{} allows clients to
publish measurements to subscriptions set up by network administrators, each
of which contains one or more queries describing requested measurements.

\PS{}'s queries allow administrators to configure both asynchronous and
synchronous data collection by restricting the type of data collected, the
devices that participate, APs or active devices to observe, and times during
which to perform measurements. Pushing queries to clients allows \PS{} to
limit the energy overhead of the measurement process, even for asynchronous
queries when the data can most likely be collected during charging sessions.
\PS{} avoids disturbing active sessions by waiting to perform asynchronous
data collection until a certain amount of time has elapsed since the last
interactive session.

To support both network monitoring and debugging over long timescales as well
as rapid network adaptation \PS{} uses both synchronous and asynchronous
queries. Asynchronous queries are configured by network administrators,
long-running, and satisfied through delay-tolerant upload. Synchronous
queries are initiated on-demand by APs, short-lived, and satisfied
immediately by participating clients. When an synchronous query arrives
\PS{} clients can decide whether they should collect and return the requested
measurements, a decision that is determined by several factors:

\begin{itemize}

\item \textbf{Usage status.} Because the goal of \PS{} is to avoid disturbing
active sessions active clients will ignore synchronous queries.

\item \textbf{Relationships between devices.} \PS{} allows users to
configure their app to always return data about other devices that they own.
Relationships between devices can be manually configured through the app,
or a list of other devices associated with a given user can be retrieved from
the \PS{} service to assist this process.

\item \textbf{Proximity to active clients.} \PS{} clients must determine
whether they can provide measurements approximating the network conditions
experience by the active clients included in the query. We return to the
challenge of proximity detection and the question of whether to perform
proximity detection in signal or physical space in
Section~\ref{sec-challenges}.

\item \textbf{Battery level.} Because \PS{} runs on energy-constrained
  devices clients are free to not participate if they are low on energy. The
  \PS{} app allows users to configure separate battery thresholds for queries
  related and unrelated to their other devices.

\end{itemize}

If the client decides to participate in the synchronous query, it performs
the measurements and sends them directly to the AP. Note that if measurements
made to satisfy synchronous queries also match existing asynchronous queries,
they will also be uploaded to the \PS{} subscription by the AP.

\subsection{Network Adaptation}

Finally, in the case of synchronous adaptation synchronous query results must
consumed by CANSAS algorithms that are capable of adjust network parameters
accordingly. For large enterprise network deployments, these algorithms might
run at a central location capable of adjusting AP channels and power levels,
similar to what is already done today albeit using only data from APs and not
from clients. For home networks smart \PS{} APs might run these algorithms
locally, or users could connect them to CANSAS algorithms running as services
in the cloud, with this last approach enabling coordination between neighbors
with overlapping wireless networks.
