\newpage
\section{PocketSniffer Design}
\label{sec-design}

This section describes the design of \PS{}, a system enabling CANSAS for Wifi
networks by providing the client-side measurements needed to enable the
coordination algorithms described in the next section. We begin with an
overview of the \PS{} system from the perspective of a network user.

\subsection{Overview}

When Alice registers for her campus WLAN she also installs the \PS{}
monitoring app to her smartphone. As she travels around campus, her
smartphone collect measurements on the health and performance of the campus
network based on queries established by network administrators, reporting
these measurements in an energy-neutral way by uploading data only when her
smartphone is charging.

In addition, as Alice and Bob sit in a campus cafe surfing the web on their
laptops---also both associated to the campus WLAN---suddenly a new source of
interference begins to disrupt their network performance. Unfortunately this
interfering terminal cannot be overheard at the access point (AP) they are
associated with, but the AP can tell that Alice's and Bob's networking
environment has degraded. At this point, it triggers the \PS{} app on Alice's
smartphone to search for a less-congested channel and may move their laptops
to a different channel, adjust its power level, or suggest they associate
with a different campus AP, all without interrupting their web surfing.

Note that in this case Bob's smartphone did not participate in the network
adaptation. Bob may not have had his device with him, may not have installed
the \PS{} app, or may not have the network hardware required to perform the
detailed channel utilization measurements necessary to enable this
adaptation. Regardless of the reason, \PS{} is able to use Alice's nearby
smartphone to estimate the impact of changes on Bob's laptop to ensure that
any spectrum adaptation benefits both devices.

Later, when Alice and Bob return home to neighboring apartments with
overlapping home wireless networks, their competing networks use
PocketSniffer measurements as inputs to perform cooperative spectrum
allocation. Despite the lack of centralized control, their two networks
jointly adapt to allocate spectrum in a way that improves performance for
both Alice and Bob.

\subsection{Performing Measurements}
\label{subsec-measurement}

PocketSniffer collects two types of measurements from terminals---scan
results and spectrum utilization information---in two different
ways---asynchronously and synchronously.

\subsubsection{Measurement types:\space} Scan results are inexpensive to
collect and provide a high-level view of the network including visible APs
and the signal strength computed from their beacon messages. Android
smartphones already perform Wifi scans at regular intervals, even while
associated with a Wifi network. Recovering this data incurs no energy
overhead for clients as long as the measurements are not uploaded while the
battery is discharging, and our recent analysis of \XXXnote{FIXME} scan
results collected from \XXXnote{FIXME} smartphones over \XXXnote{FIXME} days
has demonstrated the value of this data for network monitoring and debugging.

However, because smartphones are frequently idle, periodic scans may not
reflect either locations where users actually use their devices or locations
where their devices use the network. To better connect Wifi scans with
interactive usage and network activity, \PS{} annotates each scan with two
pieces of information: (1) whether the scan was performed during
\textit{interactive use} determined by the screen state, and (2) the
timestamp since device's last data transfer, because not every interactive
session includes Wifi usage and because smartphones perform background data
transfers with the screen disabled.

In contrast, spectrum utilization measurements are expensive to collect and
may not be possible to collect on all clients, but provide a very detailed
view of spectrum usage. The ability of Wifi chipsets to observe link-layer
signaling traffic and packets sent by other terminals varies from device to
device, and because these measurements require disabling the power-save mode
used by mobile Wifi chipsets they consume extra energy even if measurement
upload is performed in an energy-neutral way. We discuss several research
challenges caused by Wifi chipset heterogeneity in
Section~\ref{sec-challenges}.

\subsubsection{Asynchronous queries:\space} Asynchronous measurements are
used to perform network monitoring and as a replacement for expensive site
surveys in order to do spatial and temporal spectral capacity planning. For
asynchronous queries the \PS{} app allows clients to publish measurements to
a \textit{CANSAS subscription} set up by network administrators, each of
which contains one or more \textit{queries} describing requested
measurements. PocketSniffer's queries allow administrators to configure
asynchronous data collection in the following ways, restricting the type of
data collected, the devices that participate, APs to observe, and times
during which to perform measurements. Pushing queries to clients allows \PS{}
to limit the energy overhead of the measurement process, even for
asynchronous queries when the data can most likely be collected during
charging sessions.

Given that \PS{}'s goal is to avoid disturbing active terminals, care must be
taken when scheduling asynchronous measurements. A clear tradeoff exists: the
closer in time measurements are taken to periods of interactive use, the more
representative they are of the spectrum conditions at locations where
smartphones are used; at the same time, however, increasing temporal
proximity makes measurements more likely to collide with the next period of
interactive use. As a compromise, \PS{} adaptively schedules asynchronous
data collection a varying amount of time after each interactive session. 

\subsubsection{Synchronous queries:\space} To support fine-grained adaptation
to changing spectrum conditions, \PS{} also supports synchronous queries.
Unlike asynchronous queries---which are initiated by terminals, long-running,
and satisfied in a delay-tolerant fashion---synchronous queries are are
on-shot and initiated directly by APs. Synchronous queries share the
asynchronous query format but also include a list of active terminals that
\PS{} is requesting information about, which allows clients to determine
whether to respond as described below. When associated to an AP included in a
\PS{} subscription that the user has already joined, the \PS{} app will
listen for synchronous queries sent by the AP over the WLAN on a dedicated
port.

When an asynchronous query arrives \PS{} clients can decide whether they
should collect and return the requested measurements, a decision that is
determined by several factors:

\begin{itemize}

\item \textbf{Usage status.} Because the goal of \PS{} is to avoid disturbing
active sessions active clients will ignore synchronous queries.

\item \textbf{Relationships between terminals.} \PS{} allows users to
configure their app to always return data about other devices that they own.
Relationships between terminals can be manually configured through the app,
or a list of other devices associated with a given user can be retrieved from
the \PS{} service to assist this process.

\item \textbf{Proximity to active terminals.} \PS{} clients must determine
whether they can provide measurements approximating the network conditions
experience by the active clients included in the query. We return to the
challenge of proximity detection and the question of whether to perform
proximity detection in signal or physical space in
Section~\ref{sec-challenges}.

\item \textbf{Battery level.} Because \PS{} runs on energy-constrained
  devices clients are free to not participate if they decide that doing so
  would negatively affect their battery life. The \PS{} app allows users to
  configure separate battery requirements for queries related and unrelated
  to their other devices.

\end{itemize}

If the terminal decides to participate in the synchronous query, it performs
the measurements and sends them directly to the AP. Note that if measurements
made to satisfy synchronous queries also match existing asynchronous queries,
they will also be uploaded to the PocketSniffer subscription by the AP.
