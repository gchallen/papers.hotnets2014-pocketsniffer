\section{PocketSniffer Design}
\label{sec-design}

This section describes the design of PocketSniffer, a system enabling CANSAS
for Wifi wireless local area networks (WLANs). PocketSniffer addresses the
practical implementation issues of providing the client-side measurements
needed to enable the CANSAS algorithms described in the following section.
We begin with an overview of the operation of the PocketSniffer system from
the perspective of a network user.

\subsection{Overview}

When Alice and Bob register for their campus WLAN they are both required to
download and install the PocketSniffer monitoring app to their smartphones.
As they travel around campus, their smartphones collect measurements on the
health and performance of their campus network, reporting these measurements
through the PocketSniffer service to their campus service provider (SP) in an
energy-neutral way by waiting to upload measurements until their devices are
plugged in and charging.

In addition, as Alice and Bob sit in a campus cafe surfing the web on their
laptops---also both associated to the campus WLAN---suddenly a new source of
interference begins to disrupt their network performance. Unfortunately this
interfering terminal cannot be overheard at the access point (AP) they are
associated with, but the AP can tell that Alice's and Bob's networking
environment has degraded. At this point, it triggers the PocketSniffer app on
their smartphones to both confirm the interference and search available
spectrum for a less-congested channel. Based on the measurements collected
from the PocketSniffer app the AP may move their laptops to a different
channel, adjust its power level, or suggest they associate with a different
AP operated by the same campus SP, all without interrupting their web
surfing. In addition, if Alice and Bob are collocated and either one of them
does not have their smartphone, the PocketSniffer service can use the one
available smartphone to estimate spectrum availability at their shared
location. Later, when Alice and Bob return home to neighboring apartments
with overlapping home wireless networks, their competing networks use
PocketSniffer measurements as inputs to perform cooperative spectrum
allocation. Despite the lack of centralized control, their two networks
jointly adapt to allocate spectrum in a way that improves performance for
both Alice and Bob.

\subsection{Performing Measurements}
\label{subsec-measurement}

PocketSniffer collects two types of measurements from terminals---scan
results and spectrum utilization information---in two different
ways---asynchronously and synchronously.

\subsubsection{Measurement types:\space} Scan results are inexpensive to
collect and provide a high-level view of the network including visible APs
and the signal strength computed from their beacon messages. Android
smartphones already perform Wifi scans at regular intervals, even while
already associated with a given network (although at a slower rate).
Providing this data to SPs creates no energy overhead for terminals as long
as the measurements are not uploaded while the battery is discharging.
However, because smartphones are frequently idle, periodic scans may not
reflect both (1) locations where users actually use their devices and (2)
locations where their devices use the network. To better connect Wifi scans
with interactive usage and network activity, PocketSniffer annotates each
scan with two pieces of information. First, we indicate whether the scan was
performed during \textit{interactive use} determined by the screen state.
Second, we include the timestamp since device's last data transfer, because
not every interactive session includes Wifi usage and because smartphones
perform background data transfers with the screen disabled.

In contrast, spectrum utilization measurements are expensive to
collect and may not be possible to collect on all clients, but provide
a very detailed view of spectrum usage. The ability of Wifi chipsets
to observe link-layer signaling traffic and packets sent by other
terminals varies from device to device, and because these measurements
require disabling the power-save mode used by mobile Wifi chipsets
they consume extra energy even if measurement upload is performed in
an energy-neutral way.

We are exploring several ways to address the energy overhead of performing
detailed spectrum utilization measurements. One is to limit these
measurements to support synchronous adaptation by active terminals operated
by the same client: i.e., by Bob's smartphone only to help Bob's laptop
locate a better channel in the example above. In the second part of the
example, if Bob has forgotten his smartphone, Alice's device may be unwilling
to incur the battery drain necessary to help Bob and only agree to provide
scan results rather than detailed channel utilization measurements. A second
approach is to ensure that the algorithms triggering synchronous data
collection only request detailed measurements when needed. Finally, the
PocketSniffer app includes an energy cap applied over each charging session
limiting the total energy clients will devote to PocketSniffer measurements.
Once this limit is exhausted, the client will not contribute measurements in
response to synchronous queries or perform detailed measurements.

\subsubsection{Asynchronous queries:\space} Asynchronous measurements are
used to perform network monitoring and as a replacement for expensive site
surveys in order to do spatial and temporal spectral capacity planning. For
asynchronous queries the PocketSniffer app allows clients to publish
measurements to a \textit{CANSAS subscription} set up by SPs. Each CANSAS
subscription contains the set of APs operated by the SP, an endpoint where
the SP will receive data published by clients, and a list of one or more
\textit{queries} describing requested measurements. The PocketSniffer app
periodically retrieves this information from a central server and updates its
SP list and metadata. In order to verify the accuracy of these subscription
descriptions, each is signed by the issuer using a certificate issued by a
known certificate authority. To avoid flooding users with many irrelevant
subscriptions---for example, to SPs located in other states---the
PocketSniffer app uses the history of APs that the client has seen to filter
the subscription list to WLANs it can measure.

PocketSniffer's queries allow SPs to configure asynchronous data collection
in the following ways:

\begin{itemize}

\item \textbf{By type:} queries may request scans, detailed spectrum
measurements, or both.

\item \textbf{By device:} queries may limit themselves to a device or set of
devices, allowing PocketSniffer to apply different queries to devices used by
SP administrators---which might be required to provide more data---and those
used by normal WLAN users.

\item \textbf{By AP:} queries may specify that the SP is only interested in
measuring a subset of its APs.

\item \textbf{By time and time interval:} queries can specify date and time
ranges as well as request the rate at which new measurements matching the
query should be returned. If the rate is not set, PocketSniffer will not
initiate measurements but will still collect those---particularly scan
results---performed naturally.

\end{itemize}

A simple query would request scan results only without additional filters on
devices, APs, or time---this will produce continuous scan results for all APs
operated by the SP from all PocketSniffer terminals. As a second example of
more targeted data collection, SP administrators could investigate reports of
poor performance in a particular building by requesting detailed measurements
and filtering by the subset of APs serving the building. Once the problem was
addressed, the query would be deactivated.

Given that PocketSniffer's goal is to avoid disturbing active terminals, care
must be taken when scheduling asynchronous measurements. A clear tradeoff
exists: the closer in time measurements are taken to periods of interactive
use, the more representative they are of the spectrum conditions at locations
where smartphones are used; at the same time, however, increasing temporal
proximity makes measurements more likely to collide with a new period of
interactive use. As a compromise, PocketSniffer adaptively schedules
asynchronous data collection a varying amount of time after each interactive
session. 

\subsubsection{Synchronous queries:\space} To support fine-grained adaptation
to changing spectrum conditions, PocketSniffer also supports synchronous
queries. Unlike asynchronous queries---which are initiated by terminals,
long-running, and satisfied in a delay-tolerant fashion---synchronous queries
are are on-shot and initiated directly by APs. When associated to an AP
included in a PocketSniffer subscription that the user has already joined,
the PocketSniffer app will listen for synchronous queries sent by the AP over
the WLAN on a dedicated port. Synchronous queries contain the following
information:

\begin{itemize}

\item \textbf{List of active terminals:} these are terminals that the query
is requesting information about and will normally be equal to the set of
active terminals currently associated to the AP that issued the query. This
is to help terminals determine if the measurements they have collected are
relevant to the query.

\item \textbf{Time range:} a time window over which data is requested for
this query. In some cases, terminals may already have cached data to respond
to a query generated by asynchronous data collection, such as results from a
scan performed shortly before the query was issued.

\item \textbf{Data requested:} queries may request that terminals (1) perform
scans, potentially limited to a specific set of channels; (2) collect
detailed spectrum measurements from one or multiple channels; or (3) conduct
active probing of an AP on another or the same channel by associating with it
and exchanging traffic.

\end{itemize}

Based on the query PocketSniffer terminals can decide whether they should
collect and return the requested measurements, a decision that is determined
by several factors:

\begin{itemize}

\item \textbf{Usage status.} Normally active terminals will not participate
in PocketSniffer queries, since the goal of PocketSniffer is to avoid
disturbing normal usage of the WLAN.

\item \textbf{Relationships between terminals.} PocketSniffer allows users to
configure their app to always return data about other devices that they own:
for example, Bob's smartphone will always respond to queries if his laptop is
listed as one of the active terminals. Relationships between terminals can be
manually configured through the PocketSniffer app, or a list of other devices
associated with a given user can be retrieved from the PocketSniffer service
to assist this process.

\item \textbf{Proximity to active terminals.} In the case where measurements
for a related terminal are requested by the query, PocketSniffer will assume
that the smartphone is close enough to the related device to provide useful
data and proceed to perform the requested measurement. This is because we
assume that smartphones are usually close to their user and that their user
is usually close to the active terminal.

If no measurements for a related terminal are requested, however, then this
assumption does not hold and the PocketSniffer device may not be able to
observe the network conditions at the active terminals and so not have data
relevant to the query. One option would be for the PocketSniffer device to
try to estimate its proximity to devices in the list of requested terminals,
but proximity detection on smartphones is an open research
problem~\cite{searchlight-mobicom12} and likely to consume
energy---potentially more than just performing the measurement requested by
the query. Another option is for the PocketSniffer client to simply decide
whether to participate in the query based on other factors, including its
battery level as discussed next. Note, however, that this design decision
means that PocketSniffer APs should not assume that collected data is
relevant to the query that they issued.

\item \textbf{Battery level.} Because PocketSniffer runs on
  energy-constrained devices terminals are free to not participate if they
  decide that doing so would negatively affect their battery life. The
  PocketSniffer app allows users to configure separate battery requirements
  for queries related and unrelated to their other devices.

\end{itemize}

If the terminal decides to participate in the synchronous query, it performs
the measurements and sends them directly to the AP. Note that if measurements
made to satisfy synchronous queries also match existing asynchronous queries,
they will also be uploaded to the PocketSniffer subscription by the AP.
